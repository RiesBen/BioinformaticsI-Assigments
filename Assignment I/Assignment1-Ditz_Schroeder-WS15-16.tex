%\textsl{}%!TEX TS-options = --shell-escape
%!TEX TS-program = pdflatex
\documentclass[%
   10pt,              % Schriftgroesse
   ngerman,           % wird an andere Pakete weitergereicht
   a4paper,           % Seitengroesse
   DIV11,             % Textbereichsgroesse (siehe Koma Skript Dokumentation !)
]{scrartcl}%     Klassen: scrartcl, scrreprt, scrbook, article
% -------------------------------------------------------------------------

\usepackage[utf8]{inputenc} % Font Encoding, benoetigt fuer Umlaute
\usepackage[english]{babel}   % \textsl{}Spracheinstellung

\usepackage[T1]{fontenc} % T1 Schrift Encoding
\usepackage{textcomp}    % Zusatzliche Symbole (Text Companion font extension)
\usepackage{lmodern,dsfont}     % Latin Modern Schrift
\usepackage{dsfont}
%\usepackage{wasysym}
\usepackage{ulem}
\usepackage{graphicx}
\usepackage{eurosym}
%\usepackage{txfonts}
\usepackage{stmaryrd}
\usepackage{amsfonts}
\usepackage{amsmath}
\usepackage{hyperref}
\usepackage{tikz}
\usepackage{multirow}
\usepackage{listings}
\usepackage{etextools}
\usepackage{ifthen}
\usepackage{TikZ} %phylogenetischer Baum
\usetikzlibrary{calc, shapes, backgrounds} %für die Phylogenetische bäume
\usetikzlibrary{automata,arrows}
\usepackage{subfigure} 


% Definition des Headers
\usepackage{geometry}
\geometry{a4paper, top=3cm, left=3cm, right=3cm, bottom=3cm, headsep=0mm, footskip=0mm}
\renewcommand{\baselinestretch}{1.3}\normalsize

\def\header#1#2#3#4#5#6#7{\pagestyle{empty}
\noindent
\begin{minipage}[t]{0.6\textwidth}
\begin{flushleft}
\textbf{#4}\\% Fach
#6\\% Semester
Tutor: #2  % Tutor 
\end{flushleft}
\end{minipage}
\begin{minipage}[t]{0.4\textwidth}
\begin{flushright}
\points{#7}% Punktetabelle
\vspace*{0.2cm}
#5%  Names
\end{flushright}
\end{minipage}

\begin{center}
{\Large\textbf{ Blatt #1}} % Blatt

{(Abgabe am #3)} % Abgabedatum
\end{center}
}

\newenvironment{vartab}[1]
{
    \begin{tabular}{ |c@{} *{#1}{c|} } %\hline
}{
    \end{tabular}
}

\newcommand{\myformat}[1]{& #1}

\newcommand{\entry}[1]{
  \edef\result{\csvloop[\myformat]{#1}}
  \result \\ \hline
}

\newcommand{\numbers}[1]{
  \newcounter{ctra}
\setcounter{ctra}{1}
\whiledo {\value{ctra} < #1}%
{%
  \myformat{\thectra}
  \stepcounter{ctra}%
}
\myformat{\thectra}
}
\newcommand{\emptyLine}[1]{
  \newcounter{ctra1}
\setcounter{ctra}{1}
\whiledo {\value{ctra1} < #1}%
{%
  \myformat{\hspace*{0.5cm}}
  \stepcounter{ctra1}%
}
}

\newcommand{\points}[1]{
\newcounter{colmns}
\setcounter{colmns}{#1}
\stepcounter{colmns}
  \begin{vartab}{\thecolmns}
    \numbers{#1} & $\sum$\\\hline
    \emptyLine{\thecolmns}\\
  \end{vartab}
}

\begin{document}
%\header{Blatt}{Tutor}{Abgabedatum}{Vorlesung}{Bearbeiter}{Semester}{Anzahl Aufgaben}
\header{7}{Alexander Seitz}{19. October 2015}{Bioinformatics I}{\\Jonas Ditz \\\& Benjamin Schroeder}{WS 15/16}{3}

  \section*{Exercise 1 - \textsl{Smith Waterman today}}
	  The 30 years old Smith-Waterman (SW) algorithm is one of the fundamental algorithms in bioinformatics. This dynamic programming (DP) algorithm allows in a biological sense for example the detection of Domains in a protein or a Protein in a larger sequence. But the algorithm has to face the problem of complexity of $O(n\times m)$. This complexity is getting a problem when starting to compare large sequences or even whole genomes. Still today has this algorithm a high importance in bioinformatics, which can be seen in the following examples.
	  The classical example for todays usage of the Smith Waterman algorithm is the BLAST algorithm. Which uses algorithm for seed refinement. Thereby the algorithm is accelerated with vectorization process. Basically this is an improvment on the hardwarelevel of a computer, which can be adressed by the software for parallelizing the alignment. As a result, the Smith-Waterman algorithm is accelerated by a factor of 10 in comparison to the standard algorithm Novoalign (http://novocraft.com). Also an advantage in complexity mattes is constraining the algorithm around seeds from the seeding step.
	   \cite{blast}
	  Parallelizing is general a very modern topic for Smith-Waterman Algorithms. Some research groups try to accelerate the algorithm by paralelizing the algorithm via a heterogeneous platform. On this platform they use the graphics processing unit and the central processing unit for an enhanced speed of the algorithm.
	  \cite{parallel}
	  
  \section*{Exercise 2 - \textsl{DP Algorithm for finding motifs}}
  \emph{for} i = 0 \emph{to} \emph{length}(X)\\
  \indent \emph{if} X[i] == Y[0]\\
  \indent \indent Score = 0
  \indent \indent \emph{for} j = 0 \emph{to} \emph{length}(Y)\\
  \indent \indent \indent \emph{if} (X[i+j] == Y[j]) Score += 1\\
  \indent \indent if (Score > T) 
  \indent \indent \indent \emph{print}(motif found at position i)
  \indent \indent \indent i += \emph{length}(Y)
 
  \section*{Exercise 3 - \textsl{Programming assignment: Needleman-Wunsch alignment}}
  One task was to handle command line options. We wrote a new class to have a modular 
  and easy way to provide this functionality. The java package \textit{args4j} has everything
  we need. So we used that package to write our own option handler class. The code can 
  be found in \textit{CommandLineParser.java}. We did the same with gap penalties. A
  gap penalty class gives one the advantage of modular gap handling. If a gap penalties are
  needed, just reuse this class.\\
  A new member function of our NeedlemanWunsch class takes care of file writing. If a user 
  wants the resulting alignment in a FastA file, this function takes care of it.

\begin{thebibliography}{widestlabel}
	\bibitem[Li, H., Homer, N.]{blast} 	
	Li, H., Homer, N. (2010). A survey of sequence alignment algorithms for next-generation sequencing. Briefings in Bioinformatics, 11(5), 473–483. http://doi.org/10.1093/bib/bbq015
	\bibitem[Chen, B. and Xu, Y. et al]{parallel} 		
		Chen, B. and Xu, Y. and Yang, J. and Jiang, H. (2010). A New Parallel Method of Smith-Waterman Algorithm on a Heterogeneous Platform. Springer. Algorithms and Architectures for Parallel Processing.79–90
\end{thebibliography}

\end{document}