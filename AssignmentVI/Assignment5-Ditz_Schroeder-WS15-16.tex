%\textsl{}%!TEX TS-options = --shell-escape
%!TEX TS-program = pdflatex
\documentclass[%
   10pt,              % Schriftgroesse
   nenglish,           % wird an andere Pakete weitergereicht
   a4paper,           % Seitengroesse
   DIV11,             % Textbereichsgroesse (siehe Koma Skript Dokumentation !)
]{scrartcl}%     Klassen: scrartcl, scrreprt, scrbook, article
% -------------------------------------------------------------------------

\usepackage[utf8]{inputenc} % Font Encoding, benoetigt fuer Umlaute
\usepackage[english]{babel}   % \textsl{}Spracheinstellung

\usepackage[T1]{fontenc} % T1 Schrift Encoding
\usepackage{textcomp}    % Zusatzliche Symbole (Text Companion font extension)
\usepackage{lmodern,dsfont}     % Latin Modern Schrift
\usepackage{dsfont}
\usepackage{color}
%\usepackage{wasysym}
\usepackage{ulem}
\usepackage{graphicx}
\usepackage{grffile} %allows to use pngs
\usepackage{eurosym}
%\usepackage{txfonts}
\usepackage{stmaryrd}
\usepackage{amsfonts}
\usepackage{amsmath}
\usepackage{mathtools}
\usepackage{hyperref}
\usepackage{tikz}
\usepackage{multirow}
\usepackage{listings}
\usepackage{etextools}
\usepackage{ifthen}
\usepackage{cite}
%\usepackage{TikZ} %phylogenetischer Baum
%\usetikzlibrary{calc, shapes, backgrounds} %für die Phylogenetische bäume
%\usetikzlibrary{automata,arrows}
\usepackage{subfigure} 


% Definition des Headers
\usepackage{geometry}
\geometry{a4paper, top=3cm, left=3cm, right=3cm, bottom=3cm, headsep=0mm, footskip=0mm}
\renewcommand{\baselinestretch}{1.3}\normalsize

\def\header#1#2#3#4#5#6#7{\pagestyle{empty}
\noindent
\begin{minipage}[t]{0.6\textwidth}
\begin{flushleft}
\textbf{#4}\\% Fach
#6\\% Semester
Tutor: #2  % Tutor 
\end{flushleft}
\end{minipage}
\begin{minipage}[t]{0.4\textwidth}
\begin{flushright}
\points{#7}% Punktetabelle
\vspace*{0.2cm}
#5%  Names
\end{flushright}
\end{minipage}

\begin{center}
{\Large\textbf{ Assignment #1}} % Blatt

{(Handed in #3)} % Abgabedatum
\end{center}
}

\newenvironment{vartab}[1]
{
    \begin{tabular}{ |c@{} *{#1}{c|} } %\hline
}{
    \end{tabular}
}

\newcommand{\myformat}[1]{& #1}

\newcommand{\entry}[1]{
  \edef\result{\csvloop[\myformat]{#1}}
  \result \\ \hline
}

\newcommand{\numbers}[1]{
  \newcounter{ctra}
\setcounter{ctra}{1}
\whiledo {\value{ctra} < #1}%
{%
  \myformat{\thectra}
  \stepcounter{ctra}%
}
\myformat{\thectra}
}
\newcommand{\emptyLine}[1]{
  \newcounter{ctra1}
\setcounter{ctra}{1}
\whiledo {\value{ctra1} < #1}%
{%
  \myformat{\hspace*{0.5cm}}
  \stepcounter{ctra1}%
}
}

\newcommand{\points}[1]{
\newcounter{colmns}
\setcounter{colmns}{#1}
\stepcounter{colmns}
  \begin{vartab}{\thecolmns}
    \numbers{#1} & $\sum$ (6)\\\hline   %add here Complete number of points ---------------------------
    \emptyLine{\thecolmns}\\
  \end{vartab}
}


\begin{document}
%\header{Blatt}{Tutor}{Abgabedatum}{Vorlesung}{Bearbeiter}{Semester}{Anzahl Aufgaben}
\header{6}{Alexander Seitz}{23. November 2015}{Bioinformatics I}{\\Jonas Ditz \\\& Benjamin Schroeder}{WS 15/16}{2}

\section*{Theoretical Assignment - \textsl{Sequence-profile alignment and expected patterns in sequences}}

\subsection*{(a)}
The profile as a PSWM for the given MSA would look like table \ref{tab:pswm}.

\begin{table}[h]
\centering
\caption{PSWM of the given MSA}
\begin{tabular}{c|ccccc}
   & $p_1$ & $p_2$ & $p_3$ & $p_4$ & $p_5$ \\
 \hline  
 \textbf{A} & $0.\overline{3}$ & 0 & $0.\overline{3}$ & 0 & 0 \\
 \textbf{C} & 0 & 0 & 0 & 1 & 0 \\
 \textbf{G} & 0 & 0 & 0 & 0 & 1 \\
 \textbf{T} & 0 & 1 & $0.\overline{6}$ & 0 & 0 \\
 \textbf{-} & $0.\overline{6}$ & 0 & 0 & 0 & 0 
 \label{tab:pswm}
\end{tabular}
\label{tab:pswm}
\end{table}

\noindent Using this PSWM we can now compute an optimal semiglobal alignment of our profile with the 
sequence $A = CATTCCGTTC$. First we calculate the scoring matrix using as a scoring function 
$s(a,b) = -1$, $s(a,a) = 3$ and $d = 2$:

\bigskip

\begin{table}[h]
\centering
\begin{tabular}{c|cccccccccc}
   & $b_1$ & $b_2$ & $b_3$ & $b_4$ & $b_5$ & $b_6$ & $b_7$ & $b_8$ & $b_9$ & $b_{10}$ \\
   & C & A & T & T & C & C & G & T & T & C \\
 \hline
 $p_1$ & $-1.\overline{6}$ & $-0.\overline{3}$ & $-1.\overline{6}$ & $-1.\overline{6}$ & $-1.\overline{6}$ & $-1.\overline{6}$ & $-1.\overline{6}$ & $-1.\overline{6}$ & $-1.\overline{6}$ & $-1.\overline{6}$ \\
 $p_2$ & $-1$ & $-1$ & $3$ & $3$ & $-1$ & $-1$ & $-1$ & $3$ & $3$ & $-1$ \\
 $p_3$ & $-1$ & $0.\overline{3}$ & $1.\overline{6}$ & $1.\overline{6}$ & $-1$ & $-1$ & $-1$ & $1.\overline{6}$ & $1.\overline{6}$ & $-1$ \\
 $p_4$ & $3$ & $-1$ & $-1$ & $-1$ & $3$ & $3$ & $-1$ & $-1$ & $-1$ & $3$ \\
 $p_5$ & $-1$ & $-1$ & $-1$ & $-1$ & $-1$ & $-1$ & $3$ & $-1$ & $-1$ & $-1$ 
\end{tabular}
\end{table}
\newpage
\noindent Now we fill the DP matrix using that scoring matrix:
\bigskip
\begin{table}[h!]
 \centering
 \begin{tabular}{c|ccccccccccc}
    & 0 & C & A & T & T & C & C & G & T & T & C \\
  \hline
  0 & 0 & 0 & 0 & 0 & 0 & 0 & 0 & 0 & 0 & 0 & 0 \\
  $p_1$ & -2 & $-1.\overline{6}$ & \textcolor{red}{$-0.\overline{3}$} & $-1.\overline{6}$ & $-1.\overline{6}$ & $-1.\overline{6}$ & $-1.\overline{6}$ & $-1.\overline{6}$ & $-1.\overline{6}$ & $-1.\overline{6}$ & $-1.\overline{6}$ \\
  $p_2$ & -4 & $-3$ & $-2.\overline{6}$ & \textcolor{red}{$2.\overline{6}$} & $1.\overline{3}$ & $-0.\overline{6}$ & $-2.\overline{6}$ & $-2.\overline{6}$ & $1.\overline{3}$ & $1.\overline{3}$ & $-0.\overline{6}$ \\
  $p_3$ & -6 & $-5$ & $-2.\overline{6}$ & $0.\overline{6}$ & \textcolor{red}{$4.\overline{3}$} & $2.\overline{3}$ & $0.\overline{3}$ & $-1.\overline{6}$ & $-0.\overline{6}$ & $3$ & $1$ \\
  $p_4$ & -8 & $-3$ & $-4.\overline{6}$ & $-1.\overline{3}$ & $2.\overline{3}$ & \textcolor{red}{$7.\overline{3}$} & \textcolor{red}{$5.\overline{3}$} & $3.\overline{3}$ & $1.\overline{3}$ & $1$ & $6$ \\
  $p_5$ & -10 & $-9$ & $-4$ & $-3.\overline{3}$ & $0.\overline{3}$ & $5.\overline{3}$ & $6.\overline{3}$ & \textcolor{red}{$8.\overline{3}$} & $6.\overline{3}$ & $4.\overline{3}$ &  $4$ \\
 \end{tabular} 
\end{table}

\noindent One can see that the optimal alignment (colored in red) is:

\begin{table}[h]
\centering
\begin{tabular}{cccccccccc}
 C&A&T&T&C&C&G&T&T&C\\
  &$p_1$&$p_2$&$p_3$&$p_4$&-&$p_5$& & & 
\end{tabular}
\end{table}


\subsection*{(b)}
\subsubsection*{i. Compute the probability that S[1...4] contains P = GT without substitutions.}
There are three different outcomes for this result. 

\begin{align}
 p_1 &: S[1] = G\textrm{ and }S[2] =T \nonumber \\
 p_2 &: S[2] = G\textrm{ and }S[3] =T \nonumber \\
 p_3 &: S[3] = G\textrm{ and }S[4] =T \nonumber 
\end{align}

\noindent Since all $S[i]$s are independent of each other, we simple multiply all probabilities for 
each $S[i]$.

\begin{align}
 P(p_1) &= \frac{1}{4} * \frac{1}{4} * 1 * 1 = \frac{1}{16} \nonumber \\
 P(p_2) &= 1 * \frac{1}{4} * \frac{1}{4} * 1 = \frac{1}{16} \nonumber \\
 P(p_3) &= 1 * 1 * \frac{1}{4} * \frac{1}{4} = \frac{1}{16} \nonumber 
\end{align}

\noindent Where $\frac{1}{4}$ is the probability to choose G or C, respectively, and $1$ is the 
probability to choose any character from the alphabet. All three outcomes would fulfill the task, 
so we simply sum up all probabilities:

\begin{equation}
 P(\textrm{S[1...4] contains GT}) = P(p_1) + P(p_2) + P(p_3) = \frac{3}{16} \nonumber
\end{equation}

\subsubsection*{ii. Compute the probability that S[1...6] contains P = AAA with at most one substitution.}
There are three different possible outcomes for P to appear at position 1:

\begin{align}
 S[1...3] &= YAA \nonumber \\
 S[1...3] &= AYA \nonumber \\
 S[1...3] &= AAY \nonumber 
\end{align}

\noindent Each of this possibilities has the probability $\frac{1}{16}$. Since we have four different 
start positions (1,2,3 and 4) and for each position three different outcomes, the final probability 
is:

\begin{equation}
 P(\textrm{S[1...6] contains AAA}) = 4 * 3 * \frac{1}{16} = \frac{3}{4} \nonumber
\end{equation}



\section*{Theoretical Assignment - \textsl{Practice writing an introduction / background for a paper}}

%\newpage
%\bibliography{Assignment4-Ditz_Schroeder-WS15-16.bib}
%\bibliographystyle{ieeetr}
 
\end{document}
