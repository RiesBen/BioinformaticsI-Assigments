%\textsl{}%!TEX TS-options = --shell-escape
%!TEX TS-program = pdflatex
\documentclass[%
   10pt,              % Schriftgroesse
   ngerman,           % wird an andere Pakete weitergereicht
   a4paper,           % Seitengroesse
   DIV11,             % Textbereichsgroesse (siehe Koma Skript Dokumentation !)
]{scrartcl}%     Klassen: scrartcl, scrreprt, scrbook, article
% -------------------------------------------------------------------------

\usepackage[utf8]{inputenc} % Font Encoding, benoetigt fuer Umlaute
\usepackage[english]{babel}   % \textsl{}Spracheinstellung

\usepackage[T1]{fontenc} % T1 Schrift Encoding
\usepackage{textcomp}    % Zusatzliche Symbole (Text Companion font extension)
\usepackage{lmodern,dsfont}     % Latin Modern Schrift
\usepackage{dsfont}
%\usepackage{wasysym}
\usepackage{ulem}
\usepackage{graphicx}
\usepackage{eurosym}
%\usepackage{txfonts}
\usepackage{stmaryrd}
\usepackage{amsfonts}
\usepackage{amsmath}
\usepackage{amssymb}
\usepackage{hyperref}
\usepackage{tikz}
\usepackage{multirow}
\usepackage{listings}
\usepackage{etextools}
\usepackage{ifthen}
%\usepackage{TikZ} %phylogenetischer Baum
%\usetikzlibrary{calc, shapes, backgrounds} %für die Phylogenetische bäume
%\usetikzlibrary{automata,arrows}
\usepackage{caption}
\usepackage{units}
\usepackage{subcaption}

% Definition des Headers
\usepackage{geometry}
\geometry{a4paper, top=3cm, left=3cm, right=3cm, bottom=3cm, headsep=0mm, footskip=0mm}
\renewcommand{\baselinestretch}{1.3}\normalsize

\def\header#1#2#3#4#5#6#7{\pagestyle{empty}
\noindent
\begin{minipage}[t]{0.6\textwidth}
\begin{flushleft}
\textbf{#4}\\% Fach
#6\\% Semester
Tutor: #2  % Tutor 
\end{flushleft}
\end{minipage}
\begin{minipage}[t]{0.4\textwidth}
\begin{flushright}
\points{#7}% Punktetabelle
\vspace*{0.2cm}
#5%  Names
\end{flushright}
\end{minipage}

\begin{center}
{\Large\textbf{ Blatt #1}} % Blatt

{(Abgabe am #3)} % Abgabedatum
\end{center}
}

\newenvironment{vartab}[1]
{
    \begin{tabular}{ |c@{} *{#1}{c|} } %\hline
}{
    \end{tabular}
}

\newcommand{\myformat}[1]{& #1}

\newcommand{\entry}[1]{
  \edef\result{\csvloop[\myformat]{#1}}
  \result \\ \hline
}

\newcommand{\numbers}[1]{
  \newcounter{ctra}
\setcounter{ctra}{1}
\whiledo {\value{ctra} < #1}%
{%
  \myformat{\thectra}
  \stepcounter{ctra}%
}
\myformat{\thectra}
}
\newcommand{\emptyLine}[1]{
  \newcounter{ctra1}
\setcounter{ctra}{1}
\whiledo {\value{ctra1} < #1}%
{%
  \myformat{\hspace*{0.5cm}}
  \stepcounter{ctra1}%
}
}

\newcommand{\points}[1]{
\newcounter{colmns}
\setcounter{colmns}{#1}
\stepcounter{colmns}
  \begin{vartab}{\thecolmns}
    \numbers{#1} & $\sum$ (7)\\\hline
    \emptyLine{\thecolmns}\\
  \end{vartab}
}

\begin{document}
%\header{Blatt}{Tutor}{Abgabedatum}{Vorlesung}{Bearbeiter}{Semester}{Anzahl Aufgaben}
\header{11}{Alexander Seitz}{1. Februar 2016}{Bioinformatics I}{\\Jonas Ditz \\\& Benjamin Schroeder}{WS 15/16}{3}

\section{Theoretical Assignment - \textit{Coverage statistics}}
\subsection{}
\subsection{}
\subsection{}
\section{Theoretical Assignment - \textit{Application of the arrival statistic for unitigs}}

\section{Theoretical Assignment - \textit{On distances}}
\subsection{title}

Consider a tree $T$ constructed with $D$ and the four nodes $i,j,k,l \in T$. Since $D$ is ultrametric, 
the following four inequalities have to hold:
\begin{align}
 d(i,j) &\le max \left\lbrace d(i,k), d(j,k) \right\rbrace \nonumber \\
 d(i,j) &\le max \left\lbrace d(i,l), d(j,l) \right\rbrace \nonumber \\
 d(k,l) &\le max \left\lbrace d(i,k), d(i,l) \right\rbrace \nonumber \\
 d(k,l) &\le max \left\lbrace d(j,k), d(j,l) \right\rbrace \nonumber
\end{align}
W.l.o.g we can assume that $d(i,k) = d(j,k) = d(i,l) = d(j,l)$. With that assumption we can rewrite 
the inequalities as
\begin{align}
 d(i,j) + d(i,j) &\le d(i,k) + d(j,l) \nonumber \\
 d(k,l) + d(k,l) &\le d(i,l) + d(j,k) \nonumber
\end{align}
And also the following inequality is valid:
\begin{equation}
 d(i,j) + d(k,l) \le max \left\lbrace d(i,k) + d(j,l), d(i,l) + d(j,k) \right\rbrace \nonumber
\end{equation}
As one can see, that is the Four-Point-Condition (4PC) and a metric fulfill 4PC if and only if it is 
additive. So $D$ is a tree metric $\square$

To show the back direction one consider the four elements $A, B, C, D \in X$ of a taxa $X$ and the 
following distance matrix:
\begin{equation}
 D = \begin{bmatrix}
        & B & C & D \\
      A & 7 & 6 & 5 \\
      B &   & 3 & 6 \\
      C &   &   & 5
     \end{bmatrix} \nonumber
\end{equation}
One can see in the script of this lecture that $D$ is a tree metric. But the Three-Point-Condition 
(3PC) is not fulfilled as can be easily shown. To fulfill 3PC the following inequality has to be 
valid:
\begin{equation}
 d(A,B) \le max \left\lbrace d(A,C), d(B,C) \right\rbrace \nonumber
\end{equation}
but
\begin{equation}
 7 \nleq max \left\lbrace 6, 3  \right\rbrace \nonumber
\end{equation}
So $D$ does not fulfill 3PC and thus $D$ is not ultra metric. $\square$

\subsection{}

\begin{thebibliography}{20}
	
\end{thebibliography}


\end{document}