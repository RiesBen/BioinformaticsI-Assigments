\documentclass{bioinfo}
\copyrightyear{2016} \pubyear{2016}

\access{}%Advance Access Publication Date: Day Month Year}
\appnotes{}%Manuscript Category}

\begin{document}
\firstpage{1}

\subtitle{}%Database applications}

\title[POODLE]{POODLE - To the core of your data}
\author[Ditz, Schroeder]{Jonas Ditz\,$^{\text{\sfb 1,}*}$ and Benjamin Schroeder\,$^{\text{\sfb 1,}*}$} 
\address{$^{\text{\sf 1}}$Department of Informatics, Eberhard Karls University, T\"ubingen, 72076, Germany.}

\corresp{$^\ast$To whom correspondence should be addressed.}

\history{}%Received on XXXXX; revised on XXXXX; accepted on XXXXX}

\editor{}%Associate Editor: XXXXXXX}

\abstract{\textbf{Motivation:} Many life science laboratories are still using Excel files do organize their data. 
	This leads to a huge workload for maintenance as well as a inconvenient access and update routine. 
	POODLE provides an easy-to-use and powerful interface to improve the work of your lab.\\
\textbf{Results:} POODLE is a Java-based web interface, which allows intuitive access, update and 
manipulation of data.\\
\textbf{Contact:} \href{}{forename.surname@student.uni-tuebingen.de}\\
\textbf{Supplementary information:} Supplementary data are available at \textit{https://github.com/derjedi/BioinformaticsI\-Assigments/tree/master/poodle\_project}
online.}

\maketitle

\section{Introduction}

\subsection{A real world pipeline}
function study
\subsection{cloning as lab strategy}
Toolkit idea
\subsection{cloning methods}
Classsic Cloning, Quickchange, Restriction free cloning

\subsection{The resulting data}
primer
cloningVectors
proteinConstructs

\enlargethispage{12pt}

\section{Approach}

As already wrote above many life-science laboratories use Excel files for storing and working with 
data. But Excel files have a few drawbacks, which make working with them inconvenient. On of the 
biggest problems occurs, if more than one member of the laboratory tries to access the file at the 
same time. That can lead to inconsistent files. Second, searching in a Excel file is not as easy as 
it could be. Furthermore, licenses for Excel are expensive and free software is not as powerful as 
commercial ones. The reason for using Excel rather than a database system is that most life-scientist 
are not familiar with database systems and therefor chooses the GUI of Excel to work with. We are 
facing that problem by combining the functionality of a database system with an intuitive and 
easy-to-learn handling.

\begin{methods}
\section{Methods}

POODLE is build as a two-layer software. The first layer consists of the database and routines to 
automatically build and update that database. The second layer consists of the web service that is 
used to access, update and manipulate data. This is the front end layer and a user only interacts 
with that layer. One major aspect of POODLE is the in-build BLAST function.

\subsection{Database layer}

SQLite \footnote{http://www.sqlite.org/} is the database system running in the background. We chose 
this software for several reasons. First, it is free of charge. Second, and more important SQLite is 
a small and fast database system written in C. So the requirement in space is very low. Since, the 
whole database is stored just in one file SQLite also has a incredibly good performance. Besides, 
SQLite guaranties that all transactions are ACID even if a system crash or power failure occurs. So 
we have a robust storage and access of data as well as a lot of functionality provided by the database 
system. In addition to the SQLite database POODLE's software creates a BLAST database for the provided 
data, too. That database is needed for the BLAST function that comes with POODLE.

The build and update process is implemented in python. So the user does not have to have knowledge in 
SQL coding but only has to execute one python script and everything is done automatically. This routine 
ensures that the SQLite database and the BLAST database are always in the same state.

\subsection{Web service}

POODLE web service is build with Vaadin \footnote{https://vaadin.com/home}. This allows a safe 
communication between the front end and the database layer as well as a smooth and easy handling of 
the side. 

\end{methods}

\begin{figure}[!tpb]%figure1
\fboxsep=0pt\colorbox{gray}{\begin{minipage}[t]{235pt} \vbox to 100pt{\vfill\hbox to
235pt{\hfill\fontsize{24pt}{24pt}\selectfont FPO\hfill}\vfill}
\end{minipage}}
%\centerline{\includegraphics{fig01.eps}}
\caption{Caption, caption.}\label{fig:01}
\end{figure}

%\begin{figure}[!tpb]%figure2
%%\centerline{\includegraphics{fig02.eps}}
%\caption{Caption, caption.}\label{fig:02}
%\end{figure}

Text Text Text Text Text Text  Text Text Text Text Text Text Text
Text Text  Text Text Text Text Text Text.
Figure~2\vphantom{\ref{fig:02}} shows that the above method  Text
Text Text Text  Text Text Text Text Text Text  Text Text.
\citealp{Boffelli03} might want to know about  text text text text
Text Text Text Text Text Text  Text Text Text Text Text Text Text
Text Text  Text Text Text Text Text Text.
Figure~2\vphantom{\ref{fig:02}} shows that the above method  Text
Text Text Text  Text Text Text Text Text Text  Text Text.
\citealp{Boffelli03} might want to know about  text text text text
Text Text Text Text Text Text Text Text Text Text Text Text Text
Text Text  Text Text Text Text Text Text.
Figure~2\vphantom{\ref{fig:02}} shows that the above method  Text
Text Text Text  Text Text Text Text Text Text  Text Text.
\citealp{Boffelli03} might want to know about  text text text text


\subsection{Test1}

Text Text Text Text Text Text  Text Text Text Text Text Text Text
Text Text  Text Text Text Text Text Text.
Figure~2\vphantom{\ref{fig:02}} shows that the above method  Text
Text Text Text  Text Text Text Text Text Text  Text Text.
\citealp{Boffelli03} might want to know about  text text text text
Text Text Text Text Text Text  Text Text Text Text Text Text Text
Text Text  Text Text Text Text Text Text.
Figure~2\vphantom{\ref{fig:02}} shows that the above method  Text
Text Text Text  Text Text Text Text Text Text  Text Text.
\citealp{Boffelli03} might want to know about  text text text text
Text Text Text Text Text Text Text Text Text Text Text Text Text
Text Text  Text Text Text Text Text Text.
Figure~2\vphantom{\ref{fig:02}} shows that the above method  Text
Text Text Text  Text Text Text Text Text Text  Text Text.
\citealp{Boffelli03} might want to know about  text text text text





\section{Discussion}

Text Text Text Text Text Text  Text Text Text Text Text Text Text
Text Text  Text Text Text Text Text Text.
Figure~2\vphantom{\ref{fig:02}} shows that the above method  Text
Text Text Text  Text Text Text Text Text Text  Text Text.
\citealp{Boffelli03} might want to know about  text text text text
Text Text Text Text Text Text  Text Text Text Text Text Text Text
Text Text  Text Text Text Text Text Text.
Figure~2\vphantom{\ref{fig:02}} shows that the above method  Text
Text Text Text  Text Text Text Text Text Text  Text Text.
\citealp{Boffelli03} might want to know about  text text text text
Text Text Text Text Text Text Text Text Text Text.




Table~\ref{Tab:01} shows that Text Text Text Text Text  Text Text
Text Text Text Text. Figure~2\vphantom{\ref{fig:02}} shows that
the above method Text Text. Text Text Text  Text Text Text Text
Text Text. Figure~2\vphantom{\ref{fig:02}} shows that the above
method Text Text. Text Text Text  Text Text Text Text Text Text.
Figure~2\vphantom{\ref{fig:02}} shows that the above method Text
Text.









%%%%%%%%%%%%%%%%%%%%%%%%%%%%%%%%%%%%%%%%%%%%%%%%%%%%%%%%%%%%%%%%%%%%%%%%%%%%%%%%%%%%%
%
%     please remove the " % " symbol from \centerline{\includegraphics{fig01.eps}}
%     as it may ignore the figures.
%
%%%%%%%%%%%%%%%%%%%%%%%%%%%%%%%%%%%%%%%%%%%%%%%%%%%%%%%%%%%%%%%%%%%%%%%%%%%%%%%%%%%%%%






\section{Conclusion}

(Table~\ref{Tab:01}) Text Text Text Text Text Text  Text Text Text
Text Text Text Text Text Text  Text Text Text Text Text Text.
Figure~2\vphantom{\ref{fig:02}} shows that the above method  Text
Text Text Text  Text Text Text Text Text Text  Text Text.
\citealp{Boffelli03} might want to know about  text text text text
Text Text Text Text Text Text  Text Text Text Text Text Text Text
Text Text  Text Text Text Text Text Text.
Figure~2\vphantom{\ref{fig:02}} shows that the above method  Text
Text Text Text  Text Text Text Text Text Text  Text Text.
\citealp{Boffelli03} might want to know about  text text text text
Text Text Text Text Text Text Text Text Text Text Text Text Text
Text Text  Text Text Text Text Text Text.
Figure~2\vphantom{\ref{fig:02}} shows that the above method  Text
Text Text Text  Text Text Text Text Text Text  Text Text.



Text Text Text Text Text Text  Text Text Text Text Text Text Text
Text Text  Text Text Text Text Text Text.
Figure~2\vphantom{\ref{fig:02}} shows that the above method  Text
Text Text Text  Text Text Text Text Text Text  Text Text.
\citealp{Boffelli03} might want to know about  text text text text

\begin{enumerate}
\item this is item, use enumerate
\item this is item, use enumerate
\item this is item, use enumerate
\end{enumerate}

Text Text Text Text Text Text Text Text Text Text Text Text Text
Text Text Text Text Text Text Text Text.
Figure~2\vphantom{\ref{fig:02}} shows\vadjust{\pagebreak} that the
above method  Text Text Text Text Text Text Text Text Text Text
Text Text.  \citealp{Boffelli03} might want to know about text
text text text Text Text Text Text Text Text  Text Text Text Text
Text Text Text Text Text Text Text Text Text Text Text.
Figure~2\vphantom{\ref{fig:02}} shows that the above method  Text
Text Text Text Text Text Text Text Text Text  Text Text.
\citealp{Boffelli03} might want to know about text text text text
Text Text Text Text Text Text  Text Text Text Text Text Text Text
Text Text Text Text Text Text Text\break Text.


Text Text Text Text Text Text  Text Text Text Text Text Text Text
Text Text  Text Text Text Text Text Text.
Figure~2\vphantom{\ref{fig:02}} shows that the above method  Text
Text Text Text\vspace*{-10pt}


\section*{Acknowledgements}

Text Text Text Text Text Text  Text Text.  \citealp{Boffelli03} might want to know about  text
text text text\vspace*{-12pt}

\section*{Funding}

This work has been supported by the... Text Text  Text Text.\vspace*{-12pt}

%\bibliographystyle{natbib}
%\bibliographystyle{achemnat}
%\bibliographystyle{plainnat}
%\bibliographystyle{abbrv}
%\bibliographystyle{bioinformatics}
%
%\bibliographystyle{plain}
%
%\bibliography{Document}


\begin{thebibliography}{}

\bibitem[Bofelli {\it et~al}., 2000]{Boffelli03}
Bofelli,F., Name2, Name3 (2003) Article title, {\it Journal Name}, {\bf 199}, 133-154.

\bibitem[Bag {\it et~al}., 2001]{Bag01}
Bag,M., Name2, Name3 (2001) Article title, {\it Journal Name}, {\bf 99}, 33-54.

\bibitem[Yoo \textit{et~al}., 2003]{Yoo03}
Yoo,M.S. \textit{et~al}. (2003) Oxidative stress regulated genes
in nigral dopaminergic neurnol cell: correlation with the known
pathology in Parkinson's disease. \textit{Brain Res. Mol. Brain
Res.}, \textbf{110}(Suppl. 1), 76--84.

\bibitem[Lehmann, 1986]{Leh86}
Lehmann,E.L. (1986) Chapter title. \textit{Book Title}. Vol.~1, 2nd edn. Springer-Verlag, New York.

\bibitem[Crenshaw and Jones, 2003]{Cre03}
Crenshaw, B.,III, and Jones, W.B.,Jr (2003) The future of clinical
cancer management: one tumor, one chip. \textit{Bioinformatics},
doi:10.1093/bioinformatics/btn000.

\bibitem[Auhtor \textit{et~al}. (2000)]{Aut00}
Auhtor,A.B. \textit{et~al}. (2000) Chapter title. In Smith, A.C.
(ed.), \textit{Book Title}, 2nd edn. Publisher, Location, Vol. 1, pp.
???--???.

\bibitem[Bardet, 1920]{Bar20}
Bardet, G. (1920) Sur un syndrome d'obesite infantile avec
polydactylie et retinite pigmentaire (contribution a l'etude des
formes cliniques de l'obesite hypophysaire). PhD Thesis, name of
institution, Paris, France.

\end{thebibliography}
\end{document}
