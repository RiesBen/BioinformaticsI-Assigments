\documentclass{bioinfo}
\copyrightyear{2016} \pubyear{2016}

\access{}%Advance Access Publication Date: Day Month Year}
\appnotes{}%Manuscript Category}

\begin{document}
\firstpage{1}

\subtitle{}%Database applications}

\title[POODLE]{POODLE 1.0 - To the core of your data}
\author[Ditz, Schroeder]{Jonas Ditz\,$^{\text{\sfb 1,}*}$ and Benjamin Schroeder\,$^{\text{\sfb 1,}*}$} 
\address{$^{\text{\sf 1}}$Department of Informatics, Eberhard Karls University, T\"ubingen, 72076, Germany.}

\corresp{$^\ast$To whom correspondence should be addressed.}

\history{}%Received on XXXXX; revised on XXXXX; accepted on XXXXX}

\editor{}%Associate Editor: XXXXXXX}

\abstract{\textbf{Motivation:} Many life science laboratories are still using Excel files do organize their data. 
	This leads to a huge workload for maintenance as well as a inconvenient access and update routine. 
	POODLE provides an easy-to-use and powerful interface to improve the work of your lab.\\
\textbf{Results:} POODLE is a Java-based web interface, which allows intuitive access, update and 
manipulation of data.\\
\textbf{Contact:} \href{}{forename.surname@student.uni-tuebingen.de}\\
\textbf{Supplementary information:} Supplementary data are available at \textit{https://github.com/derjedi/BioinformaticsI\-Assigments/tree/master/poodle\_project}
online.}

\maketitle

\section{Introduction}

\subsection{A real world pipeline}
function study
\subsection{cloning as lab strategy}
Toolkit idea
\subsection{cloning methods}
Classsic Cloning, Quickchange, Restriction free cloning

\subsection{The resulting data}
primer
cloningVectors
proteinConstructs

\enlargethispage{12pt}

\section{Approach}

As already wrote above many life-science laboratories use Excel files for storing and working with 
data. But Excel files have a few drawbacks, which make working with them inconvenient. On of the 
biggest problems occurs, if more than one member of the laboratory tries to access the file at the 
same time. That can lead to inconsistent files. Second, searching in a Excel file is not as easy as 
it could be. Furthermore, licenses for Excel are expensive and free software is not as powerful as 
commercial ones. The reason for using Excel rather than a database system is that most life-scientist 
are not familiar with database systems and therefor chooses the GUI of Excel to work with. We are 
facing that problem by combining the functionality of a database system with an intuitive and 
easy-to-learn handling.

\begin{methods}
\section{Methods}

POODLE is build as a two-layer software. The first layer consists of the database and routines to 
automatically build and update that database. The second layer consists of the web service that is 
used to access, update and manipulate data. This is the front end layer and provides access to all 
functions for the user. Because POODLE is not just a data storage software but also comes with useful 
methods like a Blast search.

\subsection{Database layer}

SQLite\footnote{http://www.sqlite.org/} is the database system running in the background. We chose 
this software for several reasons. First, it is free of charge. Second, and more important SQLite is 
a small and fast database system written in C. So the requirement in space is very low. Since, the 
whole database is stored just in one file SQLite also has a incredibly good performance. Besides, 
SQLite guaranties that all transactions are ACID even if a system crash or power failure occurs. So 
we have a robust storage and access of data as well as a lot of functionality provided by the database 
system. In addition to the SQLite database POODLE's software creates a BLAST database for the provided 
data, too. 

The build and update process is implemented in Python. So the user does not have to have knowledge in 
SQL coding but only has to execute one python script and everything is done automatically. This routine 
ensures that the SQLite database and the BLAST database are always in the same state. It also makes 
the database a little bit safer, since changes can be reviewed before inserting them into the database.

\subsection{Web service}

POODLE web service is build with Vaadin\footnote{https://vaadin.com/home}. This allows a safe 
communication between the front end and the database layer as well as a smooth and easy handling of 
the side. Since a website developed with Vaadin is highly modular, it is easy to extend the web 
service in the future. That could be necessary, if new functionalities are required. Currently, 
there are three pages with different roles. The search page provides a formula that can be used 
to perform a simple SQL search in the local database. Therefor, different files are displayed and 
can be filled by the user. A click on the search button will automatically build a SQL query and 
results are displayed, if there are available results. The new entry page allows to insert a new 
entry into the database. It also provide a Blast search against the NCBI databases to check, whether 
there are available information about the current sequence or not. Last but not least, users can 
perform a local Blast search against their one database or a remote Blast search against the NCBI 
databases on the Blast page. 

\subsection{Blast functionality}

Our software package provides not just a database functionality for powerful and secure data storage 
and a easy-to-use interface but also a Blast (\citealp{Altschul01}) functionality. This functionality 
is implemented by a combination of Blast binaries provided by NCBI and the Biojava Api 
(\citealp{Prlic01}). There are two different Blast searches available in POODLE. On one hand, there 
is a remote Blast search on the NCBI server. And on the other hand, there is a local Blast search. 
The remote Blast search is not a fully functional copy of the NCBI web service but can used to 
perform a quick and simply search against databases provided by NCBI, e.g. Swissprot (\citealp{Doni01}). But we 
do not recommend to overuse that because too many requests may cause a blacklisting by NCBI. This 
remote Blast routine is mainly for the insertion of a new entry. The idea is to check whether there 
are information about the new sequence. The local Blast routine does everything someone would 
expect from a Blast search but against the local database, hence local Blast.

\end{methods}

\begin{figure}[!tpb]%figure1
\fboxsep=0pt\colorbox{gray}{\begin{minipage}[t]{235pt} \vbox to 100pt{\vfill\hbox to
235pt{\hfill\fontsize{24pt}{24pt}\selectfont FPO\hfill}\vfill}
\end{minipage}}
%\centerline{\includegraphics{fig01.eps}}
\caption{Caption, caption.}\label{fig:01}
\end{figure}

%\begin{figure}[!tpb]%figure2
%%\centerline{\includegraphics{fig02.eps}}
%\caption{Caption, caption.}\label{fig:02}
%\end{figure}


\section{Discussion}


\section{Conclusion}


\section*{Acknowledgements}


\section*{Funding}

This work has been supported by ... nobody. Seriously, not anybody. We know that is sad but it is hard 
to convince people giving their money away.

%\bibliographystyle{natbib}
%\bibliographystyle{achemnat}
%\bibliographystyle{plainnat}
%\bibliographystyle{abbrv}
%\bibliographystyle{bioinformatics}
%
%\bibliographystyle{plain}
%
%\bibliography{Document}


\begin{thebibliography}{}

\bibitem[Altschul {\it et~al}., 1990]{Altschul01}
Altschul, S.F., Gish, W., Miller, W., Myers, E.W., Lipman, D.J. (1990) Basic local alignment search tool., {\it J. Mol. Biol.}, {\bf 215}, 403-410.

\bibitem[Prli\'{c} {\it et~al}., 2012]{Prlic01}
Prli\'{c}, A., Yates, A., Bliven, S. E., Rose, P. W., Jacobsen, J., Troshin, P. V., ... Willis, S. (2012) BioJava: an open-source framework for bioinformatics in 2012. {\it Bioinformatics}, {\bf 28}(20), 2693-2695. http://doi.org/10.1093/bioinformatics/bts494

\bibitem[O'Donavan {\it et~al}., 2002]{Doni01}
O'Donovan, C., Martin, M.J., Gattiker, A., Gasteiger, E., Bairoch, A., Apweiler, R. (2002) High-quality protein knowledge resource: SWISS-PROT and TrEMBL {\it Brief. Bioinform.}  {\bf 3} (3), 275-284. doi: 10.1093/bib/3.3.275 

\bibitem[Bag {\it et~al}., 2001]{Bag01}
Bag,M., Name2, Name3 (2001) Article title, {\it Journal Name}, {\bf 99}, 33-54.

\bibitem[Yoo \textit{et~al}., 2003]{Yoo03}
Yoo,M.S. \textit{et~al}. (2003) Oxidative stress regulated genes
in nigral dopaminergic neurnol cell: correlation with the known
pathology in Parkinson's disease. \textit{Brain Res. Mol. Brain
Res.}, \textbf{110}(Suppl. 1), 76--84.

\bibitem[Lehmann, 1986]{Leh86}
Lehmann,E.L. (1986) Chapter title. \textit{Book Title}. Vol.~1, 2nd edn. Springer-Verlag, New York.

\bibitem[Crenshaw and Jones, 2003]{Cre03}
Crenshaw, B.,III, and Jones, W.B.,Jr (2003) The future of clinical
cancer management: one tumor, one chip. \textit{Bioinformatics},
doi:10.1093/bioinformatics/btn000.

\bibitem[Auhtor \textit{et~al}. (2000)]{Aut00}
Auhtor,A.B. \textit{et~al}. (2000) Chapter title. In Smith, A.C.
(ed.), \textit{Book Title}, 2nd edn. Publisher, Location, Vol. 1, pp.
???--???.

\bibitem[Bardet, 1920]{Bar20}
Bardet, G. (1920) Sur un syndrome d'obesite infantile avec
polydactylie et retinite pigmentaire (contribution a l'etude des
formes cliniques de l'obesite hypophysaire). PhD Thesis, name of
institution, Paris, France.

\end{thebibliography}
\end{document}
